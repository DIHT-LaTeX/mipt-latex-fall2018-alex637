\documentclass[12pt]{article}

\usepackage[utf8]{inputenc}
\usepackage[english,russian]{babel}

\title{Домашняя работа №1}
\author{Алексей Шевченко}
\date{}

\begin{document}
	\maketitle

	\begin{flushright}
	    Audi multa,\\ loquere pauca \\[20pt]
	\end{flushright}

	Это мой первый документ в системе компьютерной вёрстки \LaTeX .\\ \\
	\begin{center}
	    <<\textsf{\LARGE{Ура!!!}}>>
	\end{center} \par
	А теперь формулы. \textsc{Формула}~--- краткое и точное словесное выражение, определение или же ряд математических величин, выраженный условными знаками.\vspace{15pt} \par
	\hspace{14pt} \textbf{\large{Термодинамика}} \par 
	Уравнение Менделеева--Клапейрона~--- уравнение состояния идеального газа, имеющее вид $pV=\nu RT$, где $p$~--- давление, $V$~--- объём, занимаемый газом, $T$~--- температура газа, $\nu$~--- количество вещества газа, а $R$~--- универсальная газовая постоянная.\vspace{15pt} \par
	\hspace{14pt} \textbf{\large{Геометрия}} \hfill \textbf{\large{Планиметрия}} \par
	Для плоского треугольника со сторонами $a, b, c$ и углом $\alpha$, лежащим против стороны $a$, справедливо соотношение $$a^2=b^2 + c^2 - 2bc\cos \alpha,$$ из которого можно выразить косинус угла треугольника:
	$$\cos \alpha = \frac{b^2+c^2-a^2}{2bc}.$$
	\hspace{14pt} Пусть $p$~--- полупериметр треугольника, тогда путём несложных преобразований можно получить, что 
	$$\tg \frac{\alpha}{2}=\sqrt{\frac{(p-b)(p-c)}{p(p-a)}}.$$
	\vspace{1cm} \\ На сегодня, пожалуй, хватит\dots Удачи!
\end{document}